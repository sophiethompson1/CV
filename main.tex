\documentclass[10pt, a4paper]{article}
    \usepackage[margin=0.9cm]{geometry}
    \usepackage{array}
    \usepackage{xcolor}
    \usepackage{fontawesome}
    \usepackage{enumitem}
    \usepackage{hyperref}
    \usepackage{titlesec}
    \pagenumbering{gobble}
    \titlespacing*{\section}{0pt}{0.25\baselineskip}{0.25\baselineskip}
    \setlist[itemize]{noitemsep, topsep=0pt}
    \newcolumntype{L}{>{\raggedleft}p{0.11\textwidth}}
    \newcolumntype{R}{p{0.84\textwidth}}
    \setlength{\parindent}{0pt}
    \newcommand\vsep{\color{lightgray} \vrule width 0.5pt}
    \newcommand\sect[1]{\section*{\hspace{.05cm} \Large\sc #1}}
    \newcommand\itemizespace{\vspace{-0.65\baselineskip}}
    \newcommand\tspace{\hfill}
    \begin{document}
        \begin{center}
            \bfseries\huge\sc Sophie Thompson
        \end{center}
        \vspace{-0.5\baselineskip}
        \begin{center}
            \begin{tabular*}{0.75\textwidth}{@{\extracolsep{\fill}} ccc}
                \faPhone \ \ +44 7984 391935 &
                \faEnvelope \ \ \href{mailto:sophiet1400@gmail.com}{sophiet1400@gmail.com} &
                \faGithub \ \ \href{https://github.com/sophiethompson1}{github.com/sophiethompson1}
            \end{tabular*}
            \begin{tabular*}{0.45\textwidth}{@{\extracolsep{\fill}} cc}
                \faGlobe \ \ \href{https://sophiethompson.uk/}{sophiethompson.uk} &
                \faLinkedinSquare \ \ \href{https://www.linkedin.com/in/sophiet1/}{linkedin.com/in/sophiet1}
            \end{tabular*}
        \end{center}
        \vspace{0.05\baselineskip}
        \sect{Education}
            \begin{tabular}{L!{\vsep}R}
                2018 - 2022 &\textbf{Imperial College London} \tspace MEng Computing
                    \begin{itemize}[label=\raisebox{0.25ex}{\tiny$\bullet$}]
                        \setlength{\itemindent}{-0.125in}
                        \item Year 3: First-class honours with 82.2\% in Robotics
                        \item Year 2: First-class honours with 84\% in lab projects and 82\% in Software Engineering Design
                        \item Year 1: Upper Second class honours with 88\% in Mathematical Methods
                        \itemizespace
                    \end{itemize} \\
                2011 - 2018 & \textbf{Westcliff High School for Girls} \tspace Sixth Form
                    \begin{itemize}[label=\raisebox{0.25ex}{\tiny$\bullet$}]
                        \setlength{\itemindent}{-0.125in}
                        \item A Level: A*A*A*A* in Maths, Further Maths, Chemistry, and Physics
                        \vspace{-1.1\baselineskip}
                    \end{itemize}
            \end{tabular}
            \vspace{0.5\baselineskip}

        \sect{Work Experience}
            \begin{tabular}{L!{\vsep}R}
                2021 & {\textbf{Facebook}} \tspace Software Engineer Intern
                    \smallskip

                    During a 5 month placement I worked in the Release Engineering team on an individual project. In \textbf{Python}, using \textbf{Click}, I worked on a command line tool that aided engineers debugging. The purpose of this tool was to not be reliant upon internal infrastructure so during outages the tool would still be available to the engineers. 
                \\
                2020 & {\textbf{American Express}} \tspace Software Engineer Summer Intern 
                    \smallskip

                    In a team of 7 we worked on a microservice in MARS (Modern Accounts Receivable Service), implemented in \textbf{Java} using \textbf{Spring}. The microservice was testing using \textbf{Chaos Engineering}, we killed pods and tested that the services continued to work as usual. During my time at AmEx we completed 2 sprints, for one of them I acted as Product Owner within our team. Being Product Owner meant taking on a leader role and monitoring the team's progress and effectiveness. Throughout my sprint as Product Owner I demoed and presented our product to large groups of people.
                    \smallskip
            \end{tabular}
            \vspace{0.5\baselineskip}

        \sect{Projects}
            \begin{tabular}{L!{\vsep}R}
                2021 & {\textbf{GolfCoach}} \tspace Swift \smallskip

                    In a group of 6 we developed a mobile app for iPhones that would record the user's golf swing and give feedback based on the analysis. We used \textbf{Optical Flow} to track the movement of the golf club and Apple's \textbf{ARKit} to get a 3D skeleton model. The video would playback pausing at key stages of the swing using the skeleton to recognise those positions and give feedback on the speed of motion and the angle of the joints.
                    \smallskip
                    \vspace{0.5\baselineskip} \\

                2020 & {\textbf{Unicompare}} \tspace Python \smallskip

                    In a group of 4 we developed a webapp for prospective university students to compare universities in both social and academic aspects. The webapp was mainly \textbf{Python} using \textbf{Django} and some \textbf{HTML}, \textbf{CSS} and \textbf{Javascript}. In the webapp the user gets directed to a quiz that asks what is important to them. Universities are then ranked based off of this information, the user is given a shortlist of 5. The webapp was developed through numerous design iterations with multiple stakeholders. In this project we used \textbf{PostgreSQL} for the database and \textbf{Heroku} for deployment.
                    \smallskip
                    \vspace{0.5\baselineskip} \\
                2020 & {\textbf{WACC Compiler}} \tspace Java, ARM \smallskip

                    A compiler for the simplified language WACC. This compiler was written in \textbf{Java} and is capable of translating WACC into \textbf{ARM} assembly code that can then run, giving the desired output of the program. We expanded the basic compiler to include overloading variables and functions (based on parameters and/or return types). Optimisations include constant evaluation and control flow. 
                    \smallskip
                    \vspace{0.5\baselineskip} \\
                2019 & {\textbf{Pintos}} \tspace C \smallskip

                    During this project we built on \textbf{legacy code} to build an OS that implements a scheduler, system calls and virtual memory.
                    \smallskip
                    \vspace{0.5\baselineskip} \\

                2019 & \href{https://github.com/sophiethompsonsp/TerminalSquares}{\textbf{Terminal Squares}} \tspace Java \smallskip

                    This game you get points by making squares. I made the game with gameplay options and an option of 2 human players or playing against a custom built AI.   
                    \smallskip
                    \vspace{0.5\baselineskip} \\

                2019 & {\textbf{ARM Project}} \tspace C, ARM \smallskip

                    An assembler and dissassembler in \textbf{C},  writing some assembly code that was then run on our disassembler. Followed by an extension using a Raspberry Pi to make a binary numbers game.
                    \smallskip
                    \vspace{0.5\baselineskip} \\
            \end{tabular}
            \vspace{0.5\baselineskip}

        \sect{Skills \& Interests}
            \begin{tabular}{L!{\vsep}R}
                Programming &
                    Comfortable with Python, Java and C.
                    Some experience with Haskell, C\# and Swift.
                    \smallskip
                    Comfortable with tools such as version control and CI/CD.
                    \vspace{0.5\baselineskip} \\

                Web Based &
                    Comfortable with HTML, CSS and some experience in JavaScript.
                    \vspace{0.5\baselineskip} \\

                Ten-Pin Bowling &
                    Have competed internationally both independently and as part of Team England. In 2018 became double European Champion collecting 2 golds and a silver in the European Youth Championships in Denmark. Acted as team captain in multiple teams during my bowling career.
                    \vspace{0.5\baselineskip} \\
            \end{tabular}
    \end{document}